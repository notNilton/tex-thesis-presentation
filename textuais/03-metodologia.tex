\section{Metodologia}

\begin{frame}{Metodologia}
    \begin{itemize}
        \item Este capítulo detalha a metodologia aplicada no desenvolvimento do \textit{software} RADARE.
        \item Aborda o projeto geral, arquitetura, versionamento, ambiente de desenvolvimento e linguagens escolhidas.
        \item O foco está nas contribuições de TypeScript para o \textit{front-end}, Python para o \textit{back-end} e PostgreSQL como banco de dados.
    \end{itemize}
\end{frame}

\begin{frame}{Projeto de Desenvolvimento do RADARE}
    \begin{itemize}
        \item Metodologia de desenvolvimento ágil Scrum foi adotada para o RADARE.
        \item Scrum possibilitou desenvolvimento iterativo com \textit{sprints} focados em metas prioritárias.
        \item Revisões periódicas garantiram alinhamento das funcionalidades com os objetivos do sistema.
    \end{itemize}
\end{frame}

\begin{frame}{Requisitos Funcionais e Não Funcionais}
    \begin{itemize}
        \item Os requisitos funcionais definem as capacidades do sistema (Tabela \ref{tab:req_funcional}).
        \item Exemplo de funcionalidade: visualização em tempo real dos dados reconciliados.
        \item Os requisitos não funcionais definem critérios de qualidade, como escalabilidade, segurança e desempenho (Tabela \ref{tab:ReqNaoFuncional}).
    \end{itemize}
\end{frame}

\begin{frame}{Ambiente e Ferramentas de Desenvolvimento}
    \begin{itemize}
        \item O desenvolvimento foi realizado no \textit{Visual Studio Code} (versão 1.78) e \textit{Windows 11}.
        \item Testes realizados nos navegadores Microsoft Edge, Google Chrome e Mozilla Firefox.
        \item Versionamento feito com \textit{Git} e hospedagem no \textit{GitHub} para controle de colaboração.
    \end{itemize}
\end{frame}

\begin{frame}{Arquitetura Geral do Sistema}
    \begin{itemize}
        \item Diagrama de classes define a estrutura estática do sistema, com classes como \textbf{User} e \textbf{DataReconciliation}.
        \item Diagrama de caso de uso descreve as interações do usuário com funcionalidades do sistema, como upload de dados e visualização.
        \item Ferramenta PlantUML foi utilizada para criação e atualização dos diagramas.
    \end{itemize}
\end{frame}