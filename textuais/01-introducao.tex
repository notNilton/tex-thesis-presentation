\section{Introdução}

\begin{frame}{Contexto do Setor Industrial Brasileiro}
    \begin{itemize}
        \item O setor industrial brasileiro representa cerca de 20\% do PIB e 70\% das exportações do país.
        \item Importância da inovação e investimentos em tecnologia para a evolução do setor.
    \end{itemize}
\end{frame}

\begin{frame}{Apresentação do RADARE}
    \begin{itemize}
        \item RADARE (Reconciliation and Data Analysis in a Responsive Environment): software de reconciliação de dados para processos industriais.
        \item Objetivo do RADARE: melhorar a qualidade e confiabilidade dos dados industriais.
        \item Funcionalidades: interface gráfica intuitiva e API adaptável para diferentes aplicações industriais.
    \end{itemize}
\end{frame}

\begin{frame}{Desafios dos Dados Industriais}
    \begin{itemize}
        \item A precisão dos dados nas indústrias químicas e petroquímicas é crucial para a operação e ganhos financeiros \cite{datarecshakar}.
        \item Problema: medições imprecisas devido a erros aleatórios e sistemáticos.
        \item Solução: Reconciliação de dados utilizando equações de processo, balanço de massa e energia.
    \end{itemize}
\end{frame}

\begin{frame}{Escolha do Método de Reconciliação}
    \begin{itemize}
        \item Método escolhido: Minimização de funções multivariáveis utilizando o método dos multiplicadores de Lagrange.
        \item Vantagem: solução eficiente e adaptável para otimizar a precisão dos dados.
    \end{itemize}
\end{frame}

\begin{frame}{Vantagens do Ambiente Web para o RADARE}
    \begin{itemize}
        \item Crescimento do desenvolvimento web nos últimos anos \cite{webusage}.
        \item Benefícios do ambiente web: acesso remoto, integração via API e acessibilidade.
        \item Escolha estratégica para tornar o RADARE acessível e flexível.
    \end{itemize}
\end{frame}
