\section{Referencial Teórico}

\begin{frame}{Referencial Teórico}
    \begin{itemize}
        \item Este capítulo explora a prática da reconciliação de dados e sua aplicação no setor industrial.
        \item Discorre sobre o método de minimização de multivariáveis com multiplicadores de Lagrange.
        \item Analisa o desenvolvimento de uma plataforma web para monitoramento e manipulação de dados em tempo real.
    \end{itemize}
\end{frame}

\begin{frame}{Reconciliação de Dados}
    \begin{itemize}
        \item A reconciliação de dados assegura consistência e precisão de dados em sistemas industriais e redes de sensores \cite{datarecshakar}.
        \item Corrige discrepâncias entre valores observados e esperados, utilizando modelos e relações matemáticas predefinidas.
        \item Exemplo de aplicação: sistemas automáticos de coleta de dados, como PLCs e sistemas supervisórios \cite{plcsupervisory2021}.
    \end{itemize}
\end{frame}

\begin{frame}{Método de Lagrange}
    \begin{itemize}
        \item O método dos multiplicadores de Lagrange é usado para encontrar máximos e mínimos de funções com restrições.
        \item Essencial na otimização de processos industriais, onde há necessidade de resolver problemas com múltiplas variáveis e restrições.
        \item A função de Lagrange é definida para incorporar as restrições na função objetivo, permitindo a minimização eficiente.
    \end{itemize}
\end{frame}

\begin{frame}{Plataforma Web e Acessibilidade}
    \begin{itemize}
        \item O ambiente web facilita o monitoramento e manipulação de dados em tempo real, proporcionando acessibilidade e praticidade.
        \item APIs integradas permitem interoperabilidade e acesso remoto, ideal para monitoramento contínuo.
        \item A escolha da plataforma web destaca o RADARE em relação a outras soluções menos dinâmicas.
    \end{itemize}
\end{frame}

\begin{frame}{Sinergia entre Indústria e Internet}
    \begin{itemize}
        \item A convergência entre internet e indústria possibilita maior eficiência operacional e comunicação entre sistemas.
        \item Aplicações web conectadas permitem tratamento de dados em tempo real e integração com outras tecnologias industriais.
        \item Essa sinergia impulsiona a Indústria 4.0, promovendo ecossistemas industriais mais conectados e eficientes.
    \end{itemize}
\end{frame}
