% ===============================
% Pacotes Gerais
% ===============================
\usepackage[utf8]{inputenc} % Codificação de caracteres UTF-8
\usepackage{comment}        % Permite criar blocos de comentários no documento
\usepackage{biblatex}       % Gerenciamento de referências bibliográficas
\usepackage{textcomp}       % Suporte para símbolos adicionais de texto
\usepackage{pgf}            % Pacote para criação de gráficos em TikZ
\usepackage{courier}        % Fonte Courier para textos monoespaçados
\usepackage{expl3}          % Sintaxe LaTeX3 para criação de macros avançadas
\usepackage{threeparttable} % Suporte para notas de rodapé em tabelas
\usepackage{cancel}         % Suporte para cancelamento de termos em fórmulas
\usepackage{parskip}        % Controle de espaçamento entre parágrafos
\usepackage{empheq}         % Emolduramento de equações com opções de destaque
\usepackage{tikzsymbols}    % Pacote para inclusão de símbolos em TikZ

% ===============================
% Tema e Layout do Beamer
% ===============================
\usetheme{Madrid} % Define o tema Madrid para a apresentação

% Desativa os símbolos de navegação
\setbeamertemplate{navigation symbols}{}

% Configura o rodapé para exibir o número do slide
\setbeamertemplate{footline}[frame number]

% Configura a exibição do número da página no rodapé
\setbeamertemplate{page number in head/foot}[framenumber]

% Configuração do cabeçalho com separadores e navegação por seção
\setbeamertemplate{headline}{%
  \begin{beamercolorbox}[colsep=1.5pt]{upper separation line head}
  \end{beamercolorbox}
  \begin{beamercolorbox}{section in head/foot}
      \vskip2pt\hfill\insertsectionnavigationhorizontal{\paperwidth}{}{}\vskip2pt
  \end{beamercolorbox}%
  \begin{beamercolorbox}[colsep=1.5pt]{lower separation line head}
  \end{beamercolorbox}
}

% ===============================
% Definições de Cores Personalizadas
% ===============================
\definecolor{cvut_navy}{HTML}{0065BD}    % Azul escuro personalizado
\definecolor{cvut_blue}{HTML}{6AADE4}    % Azul claro personalizado
\definecolor{cvut_gray}{HTML}{156570}    % Cinza azulado personalizado
\definecolor{faeng_blue}{HTML}{09246F}   % Azul da FAENG
\definecolor{darkblue}{rgb}{0, 0, 0.5}   % Azul escuro genérico
\definecolor{babyblue}{rgb}{0.54, 0.81, 0.94} % Azul bebê
\definecolor{lightgreen}{HTML}{90EE90}   % Verde claro
% \sethlcolor{lightblue} % Define a cor de fundo para o comando \hl (destacado)

% ===============================
% Configurações de Cores para Elementos do Beamer
% ===============================
\setbeamercolor{section in toc}{fg=black, bg=yellow} % Cor de fundo e texto para as seções no sumário
\setbeamercolor{alerted text}{fg=cvut_blue}          % Cor do texto de alerta

% ===============================
% Comandos Personalizados
% ===============================

% Redefine o nome padrão para "Figura"
\renewcommand{\figurename}{Figura}

% Comando para criar uma caixa de destaque em torno de equações
\newcommand{\boxedeq}[2]{%
  \begin{empheq}[box={\fboxsep=6pt\fbox}]{align}
    \label{#1}#2
  \end{empheq}
}

% Comando para criar uma caixa colorida em torno de equações
\newcommand{\coloredeq}[2]{%
  \begin{empheq}[box=\colorbox{lightgreen}]{align}
    \label{#1}#2
  \end{empheq}
}

% Comando para destacar uma expressão matemática com fundo colorido
\newcommand{\highlight}[1]{%
  \colorbox{red!40}{$\displaystyle#1$}
}

% Comando para destacar texto com o comando \hl
\renewcommand<>{\hl}[1]{\only#2{\beameroriginal{\hl}}{#1}}

% Comando para criar um buffer de rolagem para códigos longos
\newcommand\scroll[4][§§]{%
  \seq_set_split:Nnn\g_inputline_seq{#1}{#4}
  \seq_map_inline:Nn\g_inputline_seq{
    \seq_gput_right:Nx\g_linebuffer_seq{##1}
    \int_compare:nT{\seq_count:N\g_linebuffer_seq>#3}{
      \seq_gpop_left:NN\g_linebuffer_seq\dummy
    }
  }
  \fbox{\begin{minipage}[t][#3\baselineskip]{#2}
    \ttfamily
    \seq_map_inline:Nn\g_linebuffer_seq{\mbox{##1}\\}
  \end{minipage}}
}

% Comando para limpar o buffer do scroll
\newcommand\clearbuf{\seq_gclear:N\g_linebuffer_seq}

% Desativa o modo de sintaxe LaTeX3
\ExplSyntaxOff
